\documentclass[12pt]{third-rep}

\usepackage{url} % typeset URL's sensibly

\usepackage{pslatex} % Use Postscript fonts

\title{CAB402 Programming Paradigms \\ \vspace{2 mm} {Quantum Computing}}
\author{Thanat Chokwijitkul n9234900}
\date{}

\usepackage{pslatex} % sets up the use of PostScript fonts

\renewcommand{\bibname}{References} % changes to references

\usepackage{listings}

\usepackage{titlesec}
\titleformat{\chapter}{\normalfont\huge\bf}{\thechapter.}{20pt}{\huge\bf}

%    Q-circuit version 2
%    Copyright (C) 2004  Steve Flammia & Bryan Eastin
%    Last modified on: 9/16/2011
%
%    This program is free software; you can redistribute it and/or modify
%    it under the terms of the GNU General Public License as published by
%    the Free Software Foundation; either version 2 of the License, or
%    (at your option) any later version.
%
%    This program is distributed in the hope that it will be useful,
%    but WITHOUT ANY WARRANTY; without even the implied warranty of
%    MERCHANTABILITY or FITNESS FOR A PARTICULAR PURPOSE.  See the
%    GNU General Public License for more details.
%
%    You should have received a copy of the GNU General Public License
%    along with this program; if not, write to the Free Software
%    Foundation, Inc., 59 Temple Place, Suite 330, Boston, MA  02111-1307  USA

% Thanks to the Xy-pic guys, Kristoffer H Rose, Ross Moore, and Daniel Müllner,
% for their help in making Qcircuit work with Xy-pic version 3.8.  
% Thanks also to Dave Clader, Andrew Childs, Rafael Possignolo, Tyson Williams,
% Sergio Boixo, Cris Moore, Jonas Anderson, and Stephan Mertens for helping us test 
% and/or develop the new version.

\usepackage{xy}
\xyoption{matrix}
\xyoption{frame}
\xyoption{arrow}
\xyoption{arc}

\usepackage{ifpdf}
\ifpdf
\else
\PackageWarningNoLine{Qcircuit}{Qcircuit is loading in Postscript mode.  The Xy-pic options ps and dvips will be loaded.  If you wish to use other Postscript drivers for Xy-pic, you must modify the code in Qcircuit.tex}
%    The following options load the drivers most commonly required to
%    get proper Postscript output from Xy-pic.  Should these fail to work,
%    try replacing the following two lines with some of the other options
%    given in the Xy-pic reference manual.
\xyoption{ps}
\xyoption{dvips}
\fi

% The following resets Xy-pic matrix alignment to the pre-3.8 default, as
% required by Qcircuit.
\entrymodifiers={!C\entrybox}

\newcommand{\bra}[1]{{\left\langle{#1}\right\vert}}
\newcommand{\ket}[1]{{\left\vert{#1}\right\rangle}}
    % Defines Dirac notation. %7/5/07 added extra braces so that the commands will work in subscripts.
\newcommand{\qw}[1][-1]{\ar @{-} [0,#1]}
    % Defines a wire that connects horizontally.  By default it connects to the object on the left of the current object.
    % WARNING: Wire commands must appear after the gate in any given entry.
\newcommand{\qwx}[1][-1]{\ar @{-} [#1,0]}
    % Defines a wire that connects vertically.  By default it connects to the object above the current object.
    % WARNING: Wire commands must appear after the gate in any given entry.
\newcommand{\cw}[1][-1]{\ar @{=} [0,#1]}
    % Defines a classical wire that connects horizontally.  By default it connects to the object on the left of the current object.
    % WARNING: Wire commands must appear after the gate in any given entry.
\newcommand{\cwx}[1][-1]{\ar @{=} [#1,0]}
    % Defines a classical wire that connects vertically.  By default it connects to the object above the current object.
    % WARNING: Wire commands must appear after the gate in any given entry.
\newcommand{\gate}[1]{*+<.6em>{#1} \POS ="i","i"+UR;"i"+UL **\dir{-};"i"+DL **\dir{-};"i"+DR **\dir{-};"i"+UR **\dir{-},"i" \qw}
    % Boxes the argument, making a gate.
\newcommand{\meter}{*=<1.8em,1.4em>{\xy ="j","j"-<.778em,.322em>;{"j"+<.778em,-.322em> \ellipse ur,_{}},"j"-<0em,.4em>;p+<.5em,.9em> **\dir{-},"j"+<2.2em,2.2em>*{},"j"-<2.2em,2.2em>*{} \endxy} \POS ="i","i"+UR;"i"+UL **\dir{-};"i"+DL **\dir{-};"i"+DR **\dir{-};"i"+UR **\dir{-},"i" \qw}
    % Inserts a measurement meter.
    % In case you're wondering, the constants .778em and .322em specify
    % one quarter of a circle with radius 1.1em.
    % The points added at + and - <2.2em,2.2em> are there to strech the
    % canvas, ensuring that the size is unaffected by erratic spacing issues
    % with the arc.
\newcommand{\measure}[1]{*+[F-:<.9em>]{#1} \qw}
    % Inserts a measurement bubble with user defined text.
\newcommand{\measuretab}[1]{*{\xy*+<.6em>{#1}="e";"e"+UL;"e"+UR **\dir{-};"e"+DR **\dir{-};"e"+DL **\dir{-};"e"+LC-<.5em,0em> **\dir{-};"e"+UL **\dir{-} \endxy} \qw}
    % Inserts a measurement tab with user defined text.
\newcommand{\measureD}[1]{*{\xy*+=<0em,.1em>{#1}="e";"e"+UR+<0em,.25em>;"e"+UL+<-.5em,.25em> **\dir{-};"e"+DL+<-.5em,-.25em> **\dir{-};"e"+DR+<0em,-.25em> **\dir{-};{"e"+UR+<0em,.25em>\ellipse^{}};"e"+C:,+(0,1)*{} \endxy} \qw}
    % Inserts a D-shaped measurement gate with user defined text.
\newcommand{\multimeasure}[2]{*+<1em,.9em>{\hphantom{#2}} \qw \POS[0,0].[#1,0];p !C *{#2},p \drop\frm<.9em>{-}}
    % Draws a multiple qubit measurement bubble starting at the current position and spanning #1 additional gates below.
    % #2 gives the label for the gate.
    % You must use an argument of the same width as #2 in \ghost for the wires to connect properly on the lower lines.
\newcommand{\multimeasureD}[2]{*+<1em,.9em>{\hphantom{#2}} \POS [0,0]="i",[0,0].[#1,0]="e",!C *{#2},"e"+UR-<.8em,0em>;"e"+UL **\dir{-};"e"+DL **\dir{-};"e"+DR+<-.8em,0em> **\dir{-};{"e"+DR+<0em,.8em>\ellipse^{}};"e"+UR+<0em,-.8em> **\dir{-};{"e"+UR-<.8em,0em>\ellipse^{}},"i" \qw}
    % Draws a multiple qubit D-shaped measurement gate starting at the current position and spanning #1 additional gates below.
    % #2 gives the label for the gate.
    % You must use an argument of the same width as #2 in \ghost for the wires to connect properly on the lower lines.
\newcommand{\control}{*!<0em,.025em>-=-<.2em>{\bullet}}
    % Inserts an unconnected control.
\newcommand{\controlo}{*+<.01em>{\xy -<.095em>*\xycircle<.19em>{} \endxy}}
    % Inserts a unconnected control-on-0.
\newcommand{\ctrl}[1]{\control \qwx[#1] \qw}
    % Inserts a control and connects it to the object #1 wires below.
\newcommand{\ctrlo}[1]{\controlo \qwx[#1] \qw}
    % Inserts a control-on-0 and connects it to the object #1 wires below.
\newcommand{\targ}{*+<.02em,.02em>{\xy ="i","i"-<.39em,0em>;"i"+<.39em,0em> **\dir{-}, "i"-<0em,.39em>;"i"+<0em,.39em> **\dir{-},"i"*\xycircle<.4em>{} \endxy} \qw}
    % Inserts a CNOT target.
\newcommand{\qswap}{*=<0em>{\times} \qw}
    % Inserts half a swap gate.
    % Must be connected to the other swap with \qwx.
\newcommand{\multigate}[2]{*+<1em,.9em>{\hphantom{#2}} \POS [0,0]="i",[0,0].[#1,0]="e",!C *{#2},"e"+UR;"e"+UL **\dir{-};"e"+DL **\dir{-};"e"+DR **\dir{-};"e"+UR **\dir{-},"i" \qw}
    % Draws a multiple qubit gate starting at the current position and spanning #1 additional gates below.
    % #2 gives the label for the gate.
    % You must use an argument of the same width as #2 in \ghost for the wires to connect properly on the lower lines.
\newcommand{\ghost}[1]{*+<1em,.9em>{\hphantom{#1}} \qw}
    % Leaves space for \multigate on wires other than the one on which \multigate appears.  Without this command wires will cross your gate.
    % #1 should match the second argument in the corresponding \multigate.
\newcommand{\push}[1]{*{#1}}
    % Inserts #1, overriding the default that causes entries to have zero size.  This command takes the place of a gate.
    % Like a gate, it must precede any wire commands.
    % \push is useful for forcing columns apart.
    % NOTE: It might be useful to know that a gate is about 1.3 times the height of its contents.  I.e. \gate{M} is 1.3em tall.
    % WARNING: \push must appear before any wire commands and may not appear in an entry with a gate or label.
\newcommand{\gategroup}[6]{\POS"#1,#2"."#3,#2"."#1,#4"."#3,#4"!C*+<#5>\frm{#6}}
    % Constructs a box or bracket enclosing the square block spanning rows #1-#3 and columns=#2-#4.
    % The block is given a margin #5/2, so #5 should be a valid length.
    % #6 can take the following arguments -- or . or _\} or ^\} or \{ or \} or _) or ^) or ( or ) where the first two options yield dashed and
    % dotted boxes respectively, and the last eight options yield bottom, top, left, and right braces of the curly or normal variety.  See the Xy-pic reference manual for more options.
    % \gategroup can appear at the end of any gate entry, but it's good form to pick either the last entry or one of the corner gates.
    % BUG: \gategroup uses the four corner gates to determine the size of the bounding box.  Other gates may stick out of that box.  See \prop.

\newcommand{\rstick}[1]{*!L!<-.5em,0em>=<0em>{#1}}
    % Centers the left side of #1 in the cell.  Intended for lining up wire labels.  Note that non-gates have default size zero.
\newcommand{\lstick}[1]{*!R!<.5em,0em>=<0em>{#1}}
    % Centers the right side of #1 in the cell.  Intended for lining up wire labels.  Note that non-gates have default size zero.
\newcommand{\ustick}[1]{*!D!<0em,-.5em>=<0em>{#1}}
    % Centers the bottom of #1 in the cell.  Intended for lining up wire labels.  Note that non-gates have default size zero.
\newcommand{\dstick}[1]{*!U!<0em,.5em>=<0em>{#1}}
    % Centers the top of #1 in the cell.  Intended for lining up wire labels.  Note that non-gates have default size zero.
\newcommand{\Qcircuit}{\xymatrix @*=<0em>}
    % Defines \Qcircuit as an \xymatrix with entries of default size 0em.
\newcommand{\link}[2]{\ar @{-} [#1,#2]}
    % Draws a wire or connecting line to the element #1 rows down and #2 columns forward.
\newcommand{\pureghost}[1]{*+<1em,.9em>{\hphantom{#1}}}
    % Same as \ghost except it omits the wire leading to the left. 

\usepackage{braket}
\usepackage{physics}
\usepackage{amsmath}

%% End of preamble, the actual document starts here

\begin{document}

\maketitle % creates the title

%% Generate contents etc
\tableofcontents
%\listoffigures
%\listoftables

%% These include the actual text

\chapter{Introduction}

The theory of computation has been extensively developed during the last few decades. Computers have provided reliable solutions for a myriad of community's seemingly unsolvable problems. Notwithstanding, various complicated problems have been continuously introduced to society as it never stops growing and becomes more complex. Even though technology nowadays has been steady advancing to approach the demands of society, whereas such steady progress has its limitation since many of those problems are intricate to model or appear to require time-intensive solutions. This phenomenon implies to the necessity of a new computing revolution since classical computation no longer has an ability to reach the increased demand. \\\\
According to Moore's law, the computational power of computers would dramatically increase approximately every two years \cite{moore}. The theory can be proven real since the size of transistors has rapidly become smaller to a few nanometres \cite{qc-info}. This results in a higher number of transistors mounted on an integrated circuit. The laws of classical physics do not function with objects with such small size. Thus, quantum computing becomes a next solution to deal with these complex problems. \\\\
Quantum computing is the revolution. Although it is a relatively new field of research, this technology has the potential to introduce the world of computing to a new stage where certain computationally intense problems can be solved with a shorter amount of processing time. One can consider quantum computing as the art of utilising all the possibilities that the laws of quantum physics contribute to solving computational problems while classical computers merely use a minuscule subset of these possibilities. However, quantum computers are not a replacement for conventional computers as quantum mechanics only improve the computing efficiency for certain types of computation. \\\\
This research report delivers a brief glimpse into the world of quantum computers and the laws of quantum mechanics applied to this entirely new type of computer. It will also introduce the fundamental concepts of the field along with exploring quantum computing in practice using Language-Integrated Quantum Operations (LIQUi$\ket{}$). This report concludes with experience evaluation in the quantum computing research in relation to the content of the unit (Programming Paradigms).

\chapter{Early History}
In the early 1980s, physicist Richard Freyman recognised that it was not possible for phenomena associated with entangled particles (quantum phenomena) to be efficiently simulated on classical computers \cite{qc}. Freyman was the first who suggested that quantum-mechanical systems might have higher computational power than conventional computers. \\\\
In the same decade, Paul Benioff proved that quantum-mechanical systems were at least as powerful as classical computers because Turing machines could be modelled with such system \cite{qc-sci}. \\\\
In 1985, David Deutsch, of Oxford, published the paper "Quantum theory, the Church-Turing principle and the universal quantum computer" \cite{universal-qc}. Deutsch claimed that the universal Turing machine has superior abilities compared with Turing machine, including random number generation, parallelism and physical system simulation with finite-dimensional state spaces. \\\\
Another of Deutsch's paper "Quantum computational networks" \cite{qc-networks} was published in 1989. The article proved that quantum circuits are as powerful as the universal Turing machine. Deutsch also introduced the first truly quantum algorithm in 1990, namely Deutsch's algorithm which later was generalised to the Deutsch-Jozsa algorithm. \\\\
In 1993, Andrew Chi-Chih Yao extended Deutsch's paper "Quantum computational networks" \cite{circuit-complexity} in the paper "Quantum circuit complexity" by addressing the complexity of quantum computation according to Deutsch's work. Yao's findings indicated quantum computing researchers to concentrate on quantum circuits instead of quantum Turing machine. \\\\
Peter Shor proposed another quantum algorithm in 1994. This algorithm uses the concept of qubit entanglement and superposition for integer factorisation \cite{intro-qc}. In principle, executing the algorithm on a quantum computer would far surpass the efficiency of all classical computers. \\\\
The University of California, Harvard University, Massachusetts Institute of Technology and IBM researchers conducted an experiment using nuclear magnetic resonance (NMR) to manipulate quantum data in liquids. The team also developed a 2-bit quantum computer with radio frequency as its input. Afterwards, a new quantum algorithm that executes on quantum computers was introduced by Lov Grover of Bell Laboratories in 1996 by the name Grover's quantum algorithm \cite{intro-qc}. \\\\
In 1998, researchers at the University Innsbruck in Austria put the idea of quantum teleportation \cite{teleport}, proposed in 1993, into practice. The theorem demonstrates the concept of entanglement and teleportation. This research is an implication for data transfer and network in the quantum system.

\begingroup
\renewcommand{\cleardoublepage}{}
\renewcommand{\clearpage}{}
\chapter{Basic Concepts}
\endgroup

\section{Qubits}
A bit is the most fundamental building block of the classical model of computer, which has a single logical value, either false or true or simply 0 or 1. In a quantum computer, the quantum bit or qubit also has two computational basis states; 0 and 1, represented by $\ket{0}$ and $\ket{1}$ respectively. However, it can be in a superposition of quantum mechanical two-state systems, meaning the qubit is both in state 0 and 1 simultaneously. A superposition of the qubit can be represented by $\alpha\ket{0}+\beta\ket{1}$ for some $\alpha$ and $\beta$ such that $|\alpha|^2+|\beta|^2=1$. \\\\
In computer system, information is represented in binary form since it is stored in the registers. For example, the non-negative numbers can be represented in binary form as
$$0,1,10,11,100,101,110,111...$$
A number of bits can determine how many configurations that binary string can represent since $2^n=y$ where $n$ is a number of bits and $y$ is a number of different configurations. For example, a three-bit binary string can represent $2^3=8$ numbers including 0 to 7. On the other hand, in quantum computers, the non-negative numbers can be represented in binary form as
$$\ket{0},\ket{1},\ket{1}\otimes\ket{0},\ket{1}\otimes\ket{1},\ket{1}\otimes\ket{0}\otimes\ket{0}...$$
In this case, an integer can be written in the form of $\ket{x_{n-1}}\otimes\ket{x_{n-2}}\otimes\ket{x_{n-3}}\otimes...\otimes\ket{x_1}\otimes\ket{x_0}$ where $\ket{x}$ is a single qubit and $x\in\{0,1\}$. Therefore, a quantum register of size three must be able to represent positive integers from 0 to 7 as the following:
$$\ket{0}\otimes\ket{0}\otimes\ket{0}\equiv\ket{000}\equiv\ket{0} \ ...\ \ket{1}\otimes\ket{1}\otimes\ket{1}\equiv\ket{111}\equiv\ket{7}$$
However, each qubit can be in both states simultaneously. A superposition of a single qubit can be denoted by $1/\sqrt[]{2}(\ket{0}+\ket{1})$. Therefore, a superposition of a quantum register of size three will be
$$\frac{1}{\sqrt[]{2}}(\ket{0}+\ket{1})\otimes\frac{1}{\sqrt[]{2}}(\ket{0}+\ket{1})\otimes\frac{1}{\sqrt[]{2}}(\ket{0}+\ket{1})$$
This can be represented in binary and decimal forms without the constant respectively as
$$\ket{000}+\ket{001}+\ket{010}+\ket{011}+\ket{100}+\ket{101}+\ket{110}+\ket{111}$$
$$\equiv\ket{0}+\ket{1}+\ket{2}+\ket{3}+\ket{4}+\ket{5}+\ket{6}+\ket{7}$$
$$\equiv\sum_{x=0}^{7}\ket{x}$$

\section{Quantum Gates}

In order to explain the concept of quantum gates, the basis states must be represented differently from the previous section. Since matrix transformations will be used to demonstrate the concept of each gate, each qubit state will be represented by any two orthogonal unit column vectors as follows: 
\[
\ket{0}=
\begin{bmatrix}
    1 \\
    0 
\end{bmatrix}
,\ket{1}=
\begin{bmatrix}
    0 \\
    1 
\end{bmatrix}
\]
In this case, the state $\ket{0}$ and $\ket{1}$ are the representations of logical zero and logical one respectively. The qubit's actual state $\ket{\Psi}$ or its superposition can be represented by
\[
\ket{\Psi}=\alpha\ket{0}+\beta\ket{1}=
\begin{bmatrix}
    \alpha \\
    \beta 
\end{bmatrix}
\]
\subsection{Quantum Gates}
Each quantum gate can be represented by a square matrix. It also needs to be unitary since being unitary will preserve the unit length of the state vector $\ket{\Psi}$ after matrix multiplication and the new state must meet the normalisation criteria $|\alpha|^2+|\beta|^2=1$. Thus, given the $2\times2$ identity matrix $I$, the output state will remain in the same state. 
\[
I=
\begin{bmatrix}
    1 & 0 \\
    0 & 1 
\end{bmatrix}
\] 
\[
I\ket{\Psi}=
\begin{bmatrix}
    1 & 0 \\
    0 & 1 
\end{bmatrix}
\begin{bmatrix}
    \alpha \\
    \beta 
\end{bmatrix}
=
\begin{bmatrix}
    \alpha \\
    \beta 
\end{bmatrix}
=\ket{\Psi}
\] \\
The matrix multiplication yields a new qubit state which is identical to the input state. This kind of quantum gate can be represented by a single wire as the following:
$$\Qcircuit @C=5em @R=.7em { & \qw}$$

\subsection{The NOT gate}
The NOT gate, represented by the negation matrix, flips its input into the opposite value. This rule indicates that if the initial state of the input is 0, and result will be 1 and vice versa. Thus, given the negation matrix $X$, each state will be inverted into its opposite value after the multiplication process.
\[
X=
\begin{bmatrix}
    0 & 1 \\
    1 & 0 
\end{bmatrix}
\] 
\[
X\ket{\Psi}=
\begin{bmatrix}
    0 & 1 \\
    1 & 0 
\end{bmatrix}
\begin{bmatrix}
    \alpha \\
    \beta 
\end{bmatrix}
=
\begin{bmatrix}
    \beta \\
    \alpha 
\end{bmatrix}
\] \\
The following diagram represents the NOT gate in the quantum circuit:
$$\Qcircuit @C=3em @R=.7em {& \gate{X} & \qw}$$ \\

\subsection{The Hadamard Gate}
This quantum gate is very important since the actual state of each qubit can be in a superposition state. Given the Hadamard matrix $H$, if the input is 0, its output will be the normalised sum of both basis states (0 and 1). In contrast, if the input is 1, the output will be the normalised difference of both basis states.
\[
H=\frac{1}{\sqrt[]{2}}
\begin{bmatrix}
    1 & 1 \\
    1 & -1 
\end{bmatrix}
\] 
\[
H\Psi=\frac{1}{\sqrt[]{2}}
\begin{bmatrix}
    1 & 1 \\
    1 & -1 
\end{bmatrix}
\begin{bmatrix}
    \beta \\
    \alpha 
\end{bmatrix}
=\frac{1}{\sqrt[]{2}}
\begin{bmatrix}
    \alpha+\beta \\
    \alpha-\beta 
\end{bmatrix}
\] \\
The following diagram represents the Hadamard gate in the quantum circuit:
$$\Qcircuit @C=3em @R=.7em {& \gate{H} & \qw}$$

\subsection{Entanglement}
Each qubit has its own quantum state. However, two or more qubits can act on one another which leads to the formation of an entangled system. When qubit states are entangled, it needs to be treated as the entire system or overall state, instead of individual quantum state. For example, given a two-bit system, it can be any integer between the range 0 and 3 inclusive as the following:
$$\ket{00}+\ket{01}+\ket{10}+\ket{11}$$
Thus, its normalised superposition can be expressed as
$$\ket{\Psi}=\alpha\ket{00}+\beta\ket{01}+\gamma\ket{10}+\delta\ket{11}$$
This rule also needs to meet the condition $|\alpha|^2+|\beta|^2+|\gamma|^2+|\delta|^2=1$. Thus, given an arbitrary set of orthogonal column vectors, each vector represents a possible quantum state, it should yield a new column vector with multi-qubit states.
\[
\ket{00}=
\begin{bmatrix}
    1 \\
    0 \\
    0 \\
    0
\end{bmatrix}
,\ket{01}=
\begin{bmatrix}
    0 \\
    1 \\
    0 \\
    0
\end{bmatrix}
,\ket{10}=
\begin{bmatrix}
    0 \\
    0 \\
    1 \\
    0
\end{bmatrix}
,\ket{11}=
\begin{bmatrix}
    0 \\
    0 \\
    0 \\
    1
\end{bmatrix}
\ thus, \ket{\Psi}=
\begin{bmatrix}
    \alpha \\
    \beta \\
    \gamma \\
    \delta
\end{bmatrix}
\]

\subsection{The Controlled-NOT Gate}
Unlike single-qubit gates such as the Hadamard gate or the NOT gate, the controlled-NOT or CNOT gate operates on two qubits by flipping the value of the second bit if the first bit is 1, but the value of the second bit remains unchanged if the first bit is 0. The following matrices illustrate how the CNOT gate operate on two qubits.
\[
CNOT=
\begin{bmatrix}
    1 & 0 & 0 & 0 \\
    0 & 1 & 0 & 0 \\
    0 & 0 & 0 & 1 \\
    0 & 0 & 1 & 0
\end{bmatrix}
\]
\[
CNOT\ket{\Psi}=
\begin{bmatrix}
    1 & 0 & 0 & 0 \\
    0 & 1 & 0 & 0 \\
    0 & 0 & 0 & 1 \\
    0 & 0 & 1 & 0
\end{bmatrix}
\begin{bmatrix}
    \alpha \\
    \beta \\
    \gamma \\
    \delta
\end{bmatrix}
=
\begin{bmatrix}
    \alpha \\
    \beta \\
    \delta \\
    \gamma
\end{bmatrix}
\] \\
In order to make the demonstration more concrete, the state can be represented with a $4\times2$ matrix representing state values of both qubits.
\[
CNOT\ket{\Psi}=
\begin{bmatrix}
    1 & 0 & 0 & 0 \\
    0 & 1 & 0 & 0 \\
    0 & 0 & 0 & 1 \\
    0 & 0 & 1 & 0
\end{bmatrix}
\begin{bmatrix}
    0 & 0 \\
    0 & 1 \\
    1 & 0 \\
    1 & 1
\end{bmatrix}
=
\begin{bmatrix}
    0 & 0 \\
    0 & 1 \\
    1 & 1 \\
    1 & 0
\end{bmatrix}
\] \\
The following diagram illustrates the CNOT gate in the quantum circuit:
$$\Qcircuit @C=2em @R=.7em {& \ctrl{1} & \targ & \qw \\ & \targ & \ctrl{-1} & \qw}$$

\section{Unitary Transformation}
As demonstrated in the quantum gates section, unitary transformations can be justified as matrix operations on vectors. This kind of quantum state manipulation can be represented as the following:
$$\ket{\Psi}\mapsto M\cdot\ket{\Psi}$$
In this case $\ket{\Psi}$ is a quantum state and $M$ is a matrix. $M$ is unitary if $M'\cdot M=I$ such that $I$ is the identity matrix. Unitary transformations are reversible and opposed to being information destructive.

\section{Measurements}
Contrary to the concept of unitary transformations, measurements are not reversible and therefore information destructive. However, this kind of operation is effective in retrieving classical information back from quantum information along with interfering or destroying the quantum state. \\\\
For example, given the quantum state $\ket{\Psi}=\alpha\ket{0}+\beta\ket{1}$, the state $\ket{0}$ with probability $\alpha^2$ and the state $\ket{1}$ with probability $\beta^2$ will be used to measure the actual state $\ket{\Psi}$. The qubit will be determined to be in one of both states until a new transformation occurs. 

\bibliography{refs} % this causes the references to be listed
\bibliographystyle{apalike} % bibliography style

%% Appendices start here
\appendix
\chapter{Example of operation}

An appendix is just like any other chapter, except that it comes after
the appendix command in the master file.

One use of an appendix is to include an example of input to the system
and the corresponding output.

One way to do this is to include, unformatted, an existing input file. 
You can do this using \verb=\verbatiminput=. In this appendix we
include a copy of the C file \textsf{hello.c} and its output file
\textsf{hello.out}. If you use this facility you should make sure that
the file which you input does not contain \texttt{TAB} characters,
since \LaTeX\ treats each \texttt{TAB} as a single space; you can use
the Unix command \texttt{expand} (see manual page) to expand tabs into
the appropriate number of spaces. 

\section{Example input and output}
\label{sec:inp-eg}
\subsection{Input}
\label{sec:input}
(Actually, this isn't input, it's the source code, but it will do as
an example)

\verbatiminput{hello.c}

\subsection{Output}
\label{sec:output}

\verbatiminput{hello.out}
\subsection{Another way to include code}
You can also use the capabilities of the \texttt{listings} package to
include sections of code, it does some keyword highlighting.

\lstinputlisting[language=C]{hello.c}

% Local Variables: 
% mode: latex
% TeX-master: "report"
% End:
\end{document}